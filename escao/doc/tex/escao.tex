
%\documentclass[a4paper,12pt]{article}
\documentclass[a4paper,10pt,twocolumn]{article}
%\usepackage[brazil]{babel}
\bibliographystyle{plain} 

\begin{document}

  \title{New Horizons in Es{\c c}{\~a}o Research}
  \author{Filipi Damasceno Vianna}
  \date{Porto Alegre, \today.}
  \maketitle

  \section{Es{\c c}{\~a}o Compiler}

    \hspace{.5 cm}
    The project matrix deals with an important step to get the firmware
    correspondent to a recording combined device. The translation to an
    algorithm, has the main element to get a compiler to these machines.
    I. e., the compiler will be a project matrix automated by a software.

    Once settled the answers (or EIE), goes to the determination of the
    numbers n, m, p. And after accomplish the correspondent graph, adding
    or not the temporal step\cite{fink}.

    The main point of this work will be when there is operating a compiler,
    preferentially in a graphical interface. It only needs to be  drawn the
    main elements (input number, outputs, feedback, memory in cascade, event
    sensors, memo latches) for the compiler start to generate the firmware
    tables. Informing the input and output events (\textsf{\textbf{EIE}})
    to be completed the programming tables.

    \begin{itemize}

      \item Is it possible the compiler be bonded to a EPROM recorder?
      \item Could be created a EPROM's \textbf{Es{\c c}{\~a}o} recorder?
      \item The compiler necessarily needs to be created in OOP object
            oriented language?
      \item How to make the SPICE Language understing the Es{\c c}{\~a}o System?

    \end{itemize}

    \subsection{New hypothesis}

    \begin{itemize}

      \item Could we use the software DIA software to genereate one file
            to be parsed by our compiler?
      \item Could our compiler acord

    \end{itemize}

  \section{Es{\c c}{\~a}o RAM}

    \hspace{.5 cm}
    Circuits integration in silicon, have reached around millions of devices
    per square centimeters, also for decreasing a lot the distance and the
    transistors, high speed are reached (better than 75ns, actually). But the
    memory still stays short, slow or to expensive. The EPROM technology
    practically stays the same.

    The computers like the IBM-PC, uses a thing to increase the speed.
    Accomplish a copy from bios firmware in ram. In cache memories, i the
    motherboard. Some time, the minors stays in a internal cache of the
    microprocessor chip. Or, it�s possible for certain moments, the firmware
    goes from EPROM to a encapsulated memory in the micro processor. So a
    great way to increase the performance of a system based in a es{\c c}{\~a}o
    architecture is substitute  the EPROM by a ram, in the moment of
    initialize the system, staying in the ram during the whole system
    functioning.

    New hypothesis are created, specifically to this job.

    \begin{itemize}

      \item Several programs can be used in ROM and copy in RAM only when
            needed to be executed?
      \item Using the \textbf{Es{\c c}{\~a}o} RAM, has the possibility of the own
            system recording values in the chip?
      \item It will provide treatment of memorized variables?
      \item ...maybe auto programming - 5th generation?
      \item Could be of individual address of bit or only byte?

    \end{itemize}

    \subsection{Discrete Microprocessor}

      \begin{itemize}

        \item If initiating the microcode table of a microprocessor, would
              be possible to create a instruction set to a \textbf{Es{\c c}{\~a}o}
              can be made a discrete microprocessor? 
        \item The application could be only didactic or allows a develop of
              a Logical Programmable Controller, with only a standard 
              hardware configuration execute several tasks with only the
              change of it is programming.
        \item It would be a recreation of the microprocessor
        \item It would be bit slice?

      \end{itemize}

      \subsubsection{Beyond the simple processors}

        \hspace{.5 cm}
        Beyond the simple processors, those who just store, pull and
        push some bits from one side to another, this new theory tends
        to show one new device, a smart device.\cite{powerg4}

        Using two devices, one ROM and one RAM, we can call the ROM as the
        \textit{Sinapce device}, because it will be used in our system
        to act as the sinapce make the connections in our brain cells.

        The ROM device should be used to make the connections between
        the \textsf{\textbf{EIE}} and the \textsf{\textbf{EII}}.



  \bibliography{escao}
\end{document}
